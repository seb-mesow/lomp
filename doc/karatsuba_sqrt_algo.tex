% !TeX spellcheck = en_US
\documentclass[%
    a4paper,%
    %twocolumn%
]{article}

\parindent0pt\relax
\parskip1em\relax

\usepackage[%
    left=1cm,%
    right=1cm,%
    top=0mm,%
    bottom=0mm%
]{geometry}

% TODO pagenumbers

\usepackage{xcolor}
\usepackage{enumitem}
\usepackage{amsmath}
\usepackage{array}
\usepackage{longtable}
\usepackage{booktabs}
\usepackage{etoolbox}

\makeatletter

\setlist{%
    itemsep=0pt,%
    parsep=0pt,%
    topsep=0pt,%
    partopsep=0pt,%
}
\setlist[itemize]{%
    label={\normalfont\bfseries\textendash},%
    labelwidth=.6em,%
    leftmargin=!%
}
\setlist[enumerate]{%
    label*={\arabic*.},%
%    labelwidth=1em,%
%    leftmargin=!%
}
%\setlist[enumerate,1]{%
%    label*={\arabic*.},%
%    labelwidth=1.5em,%
%    leftmargin=!%
%}

%\newcommand*\actualCmd{\textcolor{red}}

\DeclareMathOperator\DivRem{\texttt{DivRem}}
\DeclareMathOperator\SqrtRemWrap{\texttt{SqrtRem\_Wrap}}
\DeclareMathOperator\SqrtRemImpl{\texttt{SqrtRem\_Impl}}
\DeclareMathOperator\mpnDivRem{\texttt{mpn\_DivRem}}
\DeclareMathOperator\mpnSqrtRemImpl{\texttt{mpn\_SqrtRem\_Impl}}
\DeclareMathOperator\mpnSubN{\texttt{mpn\_sub\_n}}

% bit operators
\DeclareRobustCommand*\shl {\mathbin{\texttt{>>}}}
\DeclareRobustCommand*\shr {\mathbin{\texttt{<<}}}
\DeclareRobustCommand*\band{\mathbin{\texttt{\&}}}
\DeclareRobustCommand*\bor {\mathbin{\texttt{|}}}

\newenvironment*{algo}{% begdef
    %\newcommand*\algo@functionName{#1}%
    \newcommand*\algo@noNextParskip{%
        \vspace{-\parskip}%
    }%
    \newcommand*\algo@section{\textbf}%
    \newcommand*\Syntax{%
        \algo@section{Syntax:} % note the space
    }%
    \newcommand*\Input{%
        \algo@section{Input:}%
        \algo@noNextParskip%
    }%
    \newcommand*\PreConditions{%
        \algo@section{Preconditions:}%
        \algo@noNextParskip%
    }%
    \newcommand*\Output{%
        \algo@section{Output:}%
        \algo@noNextParskip%
    }%
    \newcommand*\PostConditions{%
        \algo@section{Postconditions:}%
        \algo@noNextParskip%
    }%
    \newcommand*\DeducedPostConditions{%
        \algo@section{Deduced Postconditions:}%
        \algo@noNextParskip%
    }%
    \newcommand*\Algorithm{%
        \algo@section{Algorithm:}%
        \algo@noNextParskip%
    }%
}{% enddef
}

\newcount\pseudocode@stepLevel
\newcounter{pseudocode@step@n}
\newcounter{pseudocode@step@i}[pseudocode@step@n]
\newcounter{pseudocode@step@ii}[pseudocode@step@i]
\newcounter{pseudocode@step@iii}[pseudocode@step@ii]
\newcounter{pseudocode@step@iv}[pseudocode@step@iii]
\newcounter{pseudocode@step@v}[pseudocode@step@iv]

\renewcommand*\thepseudocode@step@n                         {\arabic{pseudocode@step@n}.}
\renewcommand*\thepseudocode@step@i  {\thepseudocode@step@n  \arabic{pseudocode@step@i}.}
\renewcommand*\thepseudocode@step@ii {\thepseudocode@step@i  \arabic{pseudocode@step@ii}.}
\renewcommand*\thepseudocode@step@iii{\thepseudocode@step@ii \arabic{pseudocode@step@iii}.}
\renewcommand*\thepseudocode@step@iv {\thepseudocode@step@iii\arabic{pseudocode@step@iv}.}
\renewcommand*\thepseudocode@step@v  {\thepseudocode@step@iv \arabic{pseudocode@step@v}.}

\def\pseudocode@intToRoman#1{%
    \ifcase#1n%
    \or i%
    \or ii%
    \or iii%
    \or iv%
    \else v%
    \fi%
}

% TODO box for label and automatic left intendion
\newbox\pseudocode@labelBox
\newlength\pseudocode@intend
\setlength\pseudocode@intend{1em}

\newenvironment{pseudocode}
{% begdef
    \setcounter{pseudocode@step@n}{0}%
    % ##1 - indention, starting with 0
    % ##2 - statement
    % ##3 - comments on this statement
    \newcommand\statement[3]{%
        \midrule%
        \pseudocode@stepLevel##1\relax%
        \edef\pseudocode@stepLevel@roman{\pseudocode@intToRoman\pseudocode@stepLevel}%
        \refstepcounter{pseudocode@step@\pseudocode@stepLevel@roman}%
        \setbox\pseudocode@labelBox\hbox{%
            \csname thepseudocode@step@\pseudocode@stepLevel@roman\endcsname%
            \space%
        }%
        \leftskip\dimexpr\pseudocode@stepLevel\pseudocode@intend+\wd\pseudocode@labelBox\relax%
        \llap{\box\pseudocode@labelBox}%
        ##2
        &%
        ##3
        \tabularnewline%
    }%
    % ##1 - comments
    \newcommand\comment[1]{%
        \midrule%
        &%
        ##1%
        \tabularnewline%
    }%
    \begin{longtable}{%
        @{}%
            >{\raggedright\parskip1em\relax\let\\\newline}%
        p{.4\linewidth}%
            >{\raggedright\parskip1em\relax\let\\\newline}%
        p{\dimexpr.6\linewidth-2\tabcolsep\relax}%
        @{}%
    }%
    \toprule%
    \centering\bfseries Statement&\centering\bfseries Comment\tabularnewline%
    \endfirsthead%
%    \toprule%
%    \centering\bfseries Statement&\centering\bfseries Comment\tabularnewline%
    \midrule%
    \endhead%
}{% enddef
    \bottomrule%
    \end{longtable}%
}

\makeatother

\begin{document}

\title{Karatsuba Square Root Algorithm in Detail}

\author{Sebastian Mesow}

\maketitle

We implement the recursive Karatsuba Square Root Algorithm by Paul Zimmermann
as described in the following two papers:

1. https://hal.inria.fr/inria-00072854/document\\
2. https://hal.inria.fr/inria-00072113/document

Hello Hello Hello Hello Hello Hello Hello Hello Hello Hello
Hello Hello Hello Hello Hello Hello Hello Hello Hello Hello
Hello Hello Hello Hello Hello Hello Hello Hello Hello Hello
Hello Hello Hello Hello Hello Hello Hello Hello Hello Hello
Hello Hello Hello Hello Hello Hello Hello Hello Hello Hello
\begin{itemize}[topsep=-\parskip]
\item Hello Hello Hello Hello Hello
\item Hello Hello Hello Hello Hello
\item Hello Hello Hello Hello Hello
\end{itemize}
\vskip\parskip
Hello Hello Hello Hello Hello Hello Hello Hello Hello Hello
Hello Hello Hello Hello Hello Hello Hello Hello Hello Hello
Hello Hello Hello Hello Hello Hello Hello Hello Hello Hello
Hello Hello Hello Hello Hello Hello Hello Hello Hello Hello
Hello Hello Hello Hello Hello Hello Hello Hello Hello Hello


\section{Implementation Function}

\subsection{1st-level Pseudo Code}

\begin{algo}
\Syntax $S, R := \SqrtRemImpl(X)$

\Input
\begin{enumerate}
\item normalized input integer $X$
\end{enumerate}

\PreConditions
\begin{enumerate}
\item Normalization Condition
    \begin{itemize}
    \item with $m$ the minimal length of $X$
    \end{itemize}
    \begin{enumerate}
    \item $X \ge \frac{B^m}{4}$
    \item $X < B^m$
    \end{enumerate}
\end{enumerate}

\Output
\begin{enumerate}
\item square root $S$
\item square root remainder $R$
\end{enumerate}

\PostConditions
\begin{enumerate}
\item Integer Square Root Condition:
    \begin{enumerate}
    \item $S^2 \le X$
    \item $(S+1)^2 > X$
    \end{enumerate}
\item Integer Square Root Remainder Condition: $S^2 + R = X$
\end{enumerate}

\DeducedPostConditions
\begin{enumerate}
\item Upper Bound for Square Root:
    \begin{itemize}
    \item $S^2\le X \land X<B^m \implies S^2<B^m$ is equivalent to
    \end{itemize}
    \begin{enumerate}
    \item $S<B^{\frac{m}{2}}$
    \end{enumerate}
\item Upper Bounds for Square Root Remainder
    \begin{itemize}
    \item $(S+1)^2>X \iff S^2+2S+1>X \iff 2S+1>X-S^2 \iff 2S\ge X-S^2$
    \item and $S^2+R=X \iff R=X-S^2$ implies
    \end{itemize}
    \begin{enumerate}
    \item $R\le 2S$
    \end{enumerate}
    \begin{itemize}
    \item $R\le2S \land S<B^{\frac{m}{2}}$ implies
    \end{itemize}
    \begin{enumerate}[resume]
    \item $R<2B^{\frac{m}{2}}$
    \end{enumerate}
\end{enumerate}

\Algorithm
\begin{pseudocode}
\statement{0}{compute $H,L$ such that}{}
\statement{1}{$HL = N$}{}
\statement{1}{$H \ge L$}{}
\statement{1}{$X = X_{23}L^2 + X_1L+ X_0$}{}
\statement{0}{$S_p, R_p := \SqrtRemWrap(X_{23})$}{}
\statement{0}{$Q, U := \DivRem(R_p L+X_1, 2S_p)$}{}
\statement{0}{$R_t := U L + X_0 - Q^2$}{}
\statement{0}{$S_t := S_p L + Q$}{}
\statement{0}{}{correct return values}
\statement{1}{if $R_t < 0$ then}{}
\statement{2}{$R := R_t + 2S_t - 1$}{}
\statement{2}{$S := S_t - 1$}{}
\statement{1}{else}{if $R_t \ge 0$}
\statement{2}{$R := R_t$}{}
\statement{2}{$S := S_t$}{}
\end{pseudocode}
\end{algo}

\subsection{2nd-level Pseudo Code}

\begin{algo}
\Syntax $S, R := \SqrtRemImpl(X)$

\Input
\begin{enumerate}
\item normalized input number $X$
\end{enumerate}

\PreConditions
\begin{enumerate}
\item Normalization Condition
    \begin{itemize}
    \item with $m$ the minimal length of $X$
    \end{itemize}
    \begin{enumerate}
    \item $X \ge \frac{B^m}{4}$
    \item $X < B^m$
    \end{enumerate}
\end{enumerate}

\Output
\begin{enumerate}
\item square root $S$
\item square root remaninder $R$
\end{enumerate}

\PostConditions
\begin{enumerate}
\item Integer Square Root Condition:
    \begin{enumerate}
    \item $S^2 \le X$
    \item $(S+1)^2 > X$
    \end{enumerate}
\item Integer Square Root Remainder Condition: $S^2 + R = X$
\end{enumerate}

\Algorithm
\begin{pseudocode}
\statement{0}{compute $H,L$ such that}{}
\statement{1}{$HL = N$}{}
\statement{1}{$H \ge L$}{}
\statement{1}{$X = X_{23}L^2 + X_1L+ X_0$}{}
\statement{0}{$S_p, R_p := \SqrtRemWrap(X_{23})$}{}
\statement{0}{}{$Q, U := \DivRem(R_p L+X_1, 2S_p)$}
\comment{%
    it holds $R_pL + X_1 = Q\cdot 2S_p + U$\\
    it holds $Q < L$ (see research paper)\\
    it holds $U < 2S_p$
}
\statement{1}{$Q_t, U_t := \DivRem(R_p L + X_1, S_p)$}{}
\comment{it holds $R_pL+X_1 = Q_tS_p + U_t$}
\comment{it must will hold $R_p L + X_1 \stackrel{!}{=} Q\cdot 2S_p + U$}
\statement{1}{$Q := \lfloor\frac{Q_t}{2}\rfloor$}{}
\comment{%
    \begin{minipage}{\linewidth}%
    if $Q_t$ is even:
    \begin{itemize}
    \item it holds $2Q = Q_t$
    \item with $U := U_t$ it holds
    $$\begin{aligned}
    R_pL + X_1 &= Q_tS_p + U_t \\
               &= 2Q\cdot S_p + U_t \\
               &= Q\cdot 2S_p + U
    \end{aligned}$$
    \end{itemize}%
    \end{minipage}%
}
\comment{%
    \begin{minipage}{\linewidth}%
    if $Q_t$ is odd:
    \begin{itemize}
    \item it holds $2Q + 1 = Q_t$
    \item with $U := U_t + S_p$ it holds
    $$\begin{aligned}
    R_pL + X_1 &= Q_tS_p + U_t \\
               &= (2Q+1)S_p + U_t \\
               &= 2Q\cdot S_p + S_p + U_t \\
               &= Q\cdot 2S_p + U
    \end{aligned}$$%
    \end{itemize}%
    \end{minipage}%
}
\statement{1}{}{correct division remainder}
\statement{2}{if $Q_t$ is odd then}{}
\statement{3}{$U := U_t + S_p$}{}
\statement{2}{else}{if $Q_t$ is even}
\statement{3}{$U := U_t$}{}
\statement{0}{}{$R_t := U L + X_0 - Q^2$}
\statement{1}{$Q_{sqr} := Q^2$}{}
\statement{1}{$R_t := UL + X_0 - Q_{sqr}$}{}
\statement{0}{}{$S_t := S L + Q$}
\statement{1}{$S_t := S_p L + Q$}{}
%\comment{it holds $S < N$ with $N = HL$}
\statement{0}{}{correct return values}
\statement{1}{if $R_t < 0$ then}{}
\statement{2}{}{$R := R_t + 2S_t - 1$}
\statement{3}{$R_{tt} := R_t + 2S_t$}{}
\statement{3}{$R := R_{tt} - 1$}{}
\statement{2}{$S := S_t - 1$}{}
\statement{1}{else}{if $R_t \ge 0$}
\statement{2}{$R := R_t$}{}
\statement{2}{$S := S_t$}{}
\end{pseudocode}
\end{algo}

\subsection{3rd-level Pseudo Code}

\begin{algo}
\Syntax $S, rB^n+R_o := \mpnSqrtRemImpl(X, n)$

\Input
\begin{enumerate}
\item normalized input number $X$
\item the exact half of its minimal length $n$
    \begin{itemize}
    \item Thus the minimal length of $X$ must be even.
    \end{itemize}
\end{enumerate}

\PreConditions
\begin{enumerate}
\item Normalization Condition
    \begin{itemize}
    \item with $2n$ the minimal length of $X$
    \end{itemize}
    \begin{enumerate}
    \item $X \ge \frac{B^{2n}}{4}$
    \item $X < B^{2n\strut}$
    \end{enumerate}
\end{enumerate}

\Output
\begin{enumerate}
\item square root $S$
    \begin{itemize}
    \item will have length $n$
    \end{itemize}
\item square root remainder without its overflow bit $R_o$
    \begin{itemize}
    \item will have length $n$
    \end{itemize}
\item square root remainder overflow bit $r$
\end{enumerate}

\PostConditions
\begin{enumerate}
\item Integer Square Root Condition:
    \begin{enumerate}
    \item $S^2 \le X$
    \item $(S+1)^2 > X$
    \end{enumerate}
\item Integer Square Root Remainder Condition: $S^2 + rB^n + R_o = X$
\item $S$ has length $n$:
    \begin{enumerate}
    \item $S \ge 0$
    \item $S < B^n$
    \end{enumerate}
\item $R_o$ has length $n$:
    \begin{enumerate}
    \item $R_o \ge 0$
    \item $R_o < B^n$
    \end{enumerate}
\end{enumerate}

\Algorithm
\begin{pseudocode}
\statement{0}{}{%
    \begin{minipage}{\linewidth}%
    compute $H,L$ such that
    \begin{itemize}
    \item $HL = N$
    \item $H \ge L$
    \item $X = X_{23}L^2 + X_1L+ X_0$
    \end{itemize}
    \end{minipage}%
}
\statement{1}{$l := \lfloor\frac{n}{2}\rfloor$}{$L:=B^l$}
\statement{1}{$h := n - l$}{$H:=B^h$}
\comment{%
    it holds $X_0 = X\mod L$\\
    it holds $X_1 = \lfloor\frac{X}{L}\rfloor\mod L$\\
    it holds $X_{23} = \frac{X}{L^2}$
}
\comment{%
    $X_0$ has length $l$\\
    $X_1$ has length $l$\\
    $X_{23}$ has the minimal length $2h$
}
\statement{0}{}{$S_p, R_p := \SqrtRemWrap(X_{23})$}
\comment{%
    \begin{minipage}{\linewidth}
    $X_{23}$ is normalized:
    \begin{itemize}
    \item Its most significant limb is equal to that of $X$.
    \item Its minimal length $2h$ is even.
    \end{itemize}%
    \end{minipage}%
}
\statement{1}{$S_p, r_pH+R_{po} := \mpnSqrtRemImpl(X_{23}, h)$}{}
\comment{%
    $S_p$ has length $h \iff S_p < H$\\
    $R_{po}$ has length $h \iff R_{po} < H$
}
\comment{$R_p = r_pH + R_{po}$}
\statement{0}{}{$Q, U := \DivRem(R_p L+X_1, 2S_p)$}
\statement{1}{}{$Q_t, U_t := \DivRem(R_p L + X_1, S_p)$}
\comment{%
    We do not have $R_p$ at hand.
    Also $R_p$ can be $H$ and thus may have a minimal length of $h+1$,
    which would make things a bit more complicate.
    We only have at hand $r_p$ and $R_{po} < N$ seperated.
    Thus we like to execute something like $\DivRem(R_{po}L + X_1, S_p)$ .
}
\comment{\begin{minipage}{\linewidth}
    Suppose, we neglect the next subtraction.
    If $r_p = 1 \iff R_p \ge H$,
    then $R_{po} = R_p - H$ and
    $$\begin{aligned}
    Q_t, U_t &:= \DivRem\big(R_{po}L + X_1, S_p\big) \\
    &= \DivRem\big((R_p - H)L + X_1, S_p\big) \\
    \iff Q_t S_p + U_t 
    &= (R_p - H)L + X_1 \\
    &= R_p L - HL + X_1 \\
    &= R_p L + X_1 - N
    \end{aligned}$$
    which is not desired.
\end{minipage}}
\comment{\begin{minipage}{\linewidth}
    How can we manipulate the dividend $R_{po}L + X_1$,\\such that\\
    $\text{manageable side effect} + Q_t S_p + U_t = \text{manipulated}$ ?
\end{minipage}}
\comment{\begin{minipage}{\linewidth}
    Observe:\\
    If we subtract $S_pL$ from the dividend, then the quotient is smaller by $L$ (manageable side effect).
\end{minipage}}
\comment{\begin{minipage}{\linewidth}
    Observe:
    \begin{itemize}
    \item Because $R_p \le 2S_p = S_p + S_p$ and $S_p < H$, is
          $$\begin{aligned}
              & R_p < S_p + H \\
          \iff& R_p - H < S_p \\
          \iff& R_p - H - S_p = R_{po} - S_p < 0
          \end{aligned}$$
    \item At this subtraction occurs an underflow ($+H$).
          Thus $R_{po} - S_p$ is in reality
          $R_p - H - S_p + H = R_p - S_p$ .
    \end{itemize}
\end{minipage}}
\comment{\begin{minipage}{\linewidth}
    Conclusion:
    \begin{itemize}
    \item If $R_p \ge H\iff r_p = 1$, then we replace $R_{po}$ by $R_{po} - S_p$
          and the \emph{obtained} quotient $q_tL+Q_{to}$ is smaller than
          the \emph{required} quotient $Q_t$ by $L$ .
    \item The $+H$ by the underflow propagates a borrow bit to the left
          and -- as a nice side effect -- mathematical precisely cancels $r_p$ .
    \end{itemize}
\end{minipage}}
\comment{
    Check of this Idea:
    $$\begin{aligned}
     q_{t,ob}L+Q_{to}, U_t
    &:= \DivRem\big((R_p - S_p)L + X_1, S_p\big) \\
    \iff (q_{t,ob}L+Q_{to})S_p + U_t
    &= (R_p - S_p)L + X_1 \\
    &= R_pL - S_pL + X_1 \\
    \iff (q_{t,ob}L+Q_{to})S_p + S_pL+ U_t &= R_pL + X_1 \\
    = \big((q_{t,ob}+1)L+Q_{to}\big)S_p + U_t \\
    \iff (q_{t,ob}+1)L+Q_{to}, U_t &=  \DivRem(R_pL + X_1, S_p) \\
    &=: q_tL+Q_{to}, U_t
    \end{aligned}$$
}
\statement{2}{if $r_p > 0$ then}{}
\statement{3}{$R_{po} := R_{po} - S_p$}{actually we would must also $r_p := 0$}
\comment{$R_{po}L + X_1$ has length $h + l = n$}
\statement{2}{$q_{t,ob}L+Q_{to}, U_t := \mpnDivRem(R_{po}L + X_1, S_p)$}{}
\comment{\begin{minipage}{\linewidth}%
    according to the postconditions of $\mpnDivRem()$
    \begin{itemize}
    \item $Q_{to} < L$, because $Q_{to}$ has length $l = n - h = (h + l) - h$
    \item $U_t < S_p \implies U_t < H$, because $U_t$ has length $h$\\(same as the divisor $S_p$)
    \item $q_{t,ob} = 0\text{ or }1$
    \end{itemize}
\end{minipage}}
\comment{%
    If $r_p > 0 \iff r_p = 1 \iff R_p \ge H$,
    then the \emph{required} $q_t$ is bigger than the \emph{obtained} $q_{t,ob}$ by $1$ .
}
\statement{2}{$q_t := q_{t,ob} + r_p$}{%
    To reflect this in reality we, just add $r_p = 1$ to $q_{t,ob}$ .\\
    If $r_p = 0 \iff R_p = R_{po}$, then the addition is useless.
}
\comment{memorize $Q_t = q_tL + Q_{to}$}
\comment{memorize $q_t = 0\text{ or }1\text{ or }2$}
\comment{%
    it holds
    $$\begin{aligned}
    Q_tS_p + U_t &= R_pL + X_1 \\
    \iff (q_tL + Q_{to})S_p + U_t &= (r_pH + R_{po})L + X_1
    \end{aligned}$$
}
\statement{1}{}{$Q := \lfloor\frac{Q_t}{2}\rfloor$}
\statement{2}{$Q_o := Q_{to}\ \texttt{>>}\ 1$}{%
    $Q_o$ has length $l$, because $Q_{to}$ has length $l$ .%
}
\statement{2}{$Q_o[l-1] := \big((q_t \band 1) \shr (b-1)\big) \bor Q_o[l-1]$}{%
    The least significant bit of $q_t$ as it shifts right floats
    into the most significant bit of the last limb of $Q_o$.\\
    Because $Q_o$ has length $l$, the last limb is at index $l-1$. (zero-based index)
}
\statement{2}{$q := q_t\ \texttt{>>}\ 1$}{%
    memorize $q = 0\text{ or }1$ .
}
\comment{memorize $Q = qL + Q_o$}
\statement{1}{}{correct division remainder}
\statement{2}{if $Q_{to} \band 1 = 1$ then}{if $Q_t$ is odd then}
\statement{3}{$uH+U_o := U_t + S_p$}{$U := U_t + S_p$}
\statement{2}{else}{if $Q_t$ is even $\iff$ if $Q_{to} \band 1 = 0$}
\statement{3}{}{$U := U_t$}
\statement{4}{$U_o := U_t$}{}
\statement{4}{$u := 0$}{}
\comment{$U_o$ has length $h$ $\iff$ $U_o < H$}
\statement{0}{}{$R_t := UL + X_0 - Q^2$}
\statement{1}{}{$Q_{sqr} := Q^2$}
\comment{%
    Recap $Q \le L$ (see research paper)\\
    Thus either $q = 0$ or $Q_o = 0$ or both $= 0$, but never together $> 0$ .\\
    Thus always $qQ_o = 0$ .
}
\comment{%
    $$\begin{aligned}
    Q^2 &= (qL+Q_o)^2 \\
    &= qL^2 + 2qLQ_o + Q_o^2 \\
    &= qL^2            + Q_o^2 \\
    &=: qL^2 + Q_{o,sqr}
    \end{aligned}$$
    See how $q$ remains the same in $Q^2$ .
}
\statement{2}{$Q_{o,sqr} := Q_o^2$}{}
\comment{%
    $Q_{o,sqr}$ has length $2l$, because $Q_o$ has length $l$ .\\
    $Q_o < L \iff Q_o^2 < L^2$
}
\comment{Memorize $Q_{sqr} = qL^2 + Q_{o,sqr}$}
\statement{1}{}{$R_t := UL + X_0 - Q_{sqr}$}
\comment{$U_oL + X_0$ has length $n = h + l$ .}
\comment{%
    $$\begin{aligned}
    R_t & = (UL+X_0)-Q_{sqr} \\ 
    = r_tN + R_{to} &:= (uH + U_o)L + X_0 - qL^2 - Q_{o,sqr} \\
    &= uHL - qL^2 +  U_oL + X_0    - Q_{o,sqr} \\
    &= uN  - qL^2 + (U_oL + X_0)   - Q_{o,sqr}
    \end{aligned}$$
}
\statement{2}{$sub\_borrow, R_{to} := \mpnSubN(U_oL + X_0,\;Q_{o,sqr},\;2l)$}{%
    $\mpnSubN(X,Y,n)$ subtracts the $n$ least significant limbs of $X$ and $Y$.\\
    It leaves other limbs untouched. Thus a borrow is still returned.
}
\comment{currently $R_{to}$ has length $2l$}
\statement{2}{}{assign borrows to their correct place}
\comment{$r_t$ is has the meaning of a carry, but can be negative.}
\statement{3}{if $h = l$ then}{$\iff$ if $n$ is even}
\comment{%
    $U_oL + X_0$ has length $h + l = n$ .\\
    $Q_{o,sqr}$ has  length $2l = n$ .\\
    Thus in this case $U_oL+X_1$ and $Q_{o,sqr}$ have the same length.\\
    Thus in this case $sub\_borrow$ goes into the carry $r_t$ .\\
    (Thus not $n$ but $2l$ is provided to $\mpnSubN()$ .)
}
\comment{%
    Because $R_{to}$ has length $2l$ , $-qL^2$ also goes into the carry $r_t$ .
}
\statement{4}{$r_t := u - sub\_borrow - q$}{}
\statement{3}{else}{if $l + 1 = h$ $\iff$ if $n\text{ is odd}$}
%\comment{thus $2l + 1 = n$}
\comment{\begin{minipage}{\linewidth}%
    $U_oL + X_1$ has length $h + l = n = 2l + 1$ .\\
    $Q_{o,sqr}$ has length $2l$ .\\
    So $U_oL + X_1$ has one limb more than $Q_{o,sqr}$.\\
    Thus to complete the subtraction $(U_oL + X_0) - Q_{o,sqr}$ the $sub\_borrow$ must borrow
    from the last (and still untouched) limb of $U_oL+X_0$
    forming the last limb of $R_{to}$ .\\
    This may delivers the actual borrow.
\end{minipage}}
\comment{%
    Because $R_{to}$ has length $2l$ ,
    $-qL^2$ also goes into the last (and still untouched) limb of $U_oL+X_0$
    forming the last limb of $R_{to}$ .%
}
\statement{4}{$actual\_borrow, R_{to}[2l] := (U_oL+X_0)[2l] - sub\_borrow - q$}{%
    The last limb of $U_oL+X_0$ is at index $2l$ . (zero-based index)%
}
\statement{4}{$r_t := u - actual\_borrow$}{}
\comment{%
    Now if $r_t$ is negative, then this stands therefore,
    that $R_t = (UL+X_o) - Q^2$ is negative.%
}
\comment{finally $R_{to}$ has a length of $n$}
\statement{0}{}{$S_t := SL + Q$}
\statement{1}{}{%
    $$\begin{aligned}
    S_t &:= S_p L + Q \\
    = s_tN + S_{to} &:= S_pL + qL + Q_o \\
    = s_tN + S_{to,high}L + Q_o  &:= (S_p + q)L + Q_o
    \end{aligned}$$%
}
\statement{2}{$s_tH + S_{to,high} := S_p +q$}{Note: $N = HL \iff n = h+l$}
\comment{$S_{o,high}$ as length $h$, because $S_p$ has length $h$ .}
\statement{2}{$S_{to} := S_{to,high}L + Q_o$}{}
\comment{$S_{to}$ as length $n = h + l$, because $S_{to,high}$ has length $h$ and $Q_o$ has length $l$.}
\comment{%
    We can only execute these statements after 3.3.1.1. , because until including 3.3.1.1.
    $S_p$ (and not $S_p + q$) is still required
    and $S_p$ is stored at memory, which will become $S_{to}$ here.
}
\statement{0}{}{correct return values}
\statement{1}{if $r_t < 0$ then}{if $R_t < 0$ then}
\statement{2}{}{$R := R_t + 2S_t - 1$}
\statement{3}{}{$R_{tt} := R_t + 2S_t$}
\statement{4}{$r_{tt}N+R_{tto} := R_{to} + 2S_{to}$}{$R_{tto}$ has length $n$ .}
\statement{4}{$r_{tt} := r_{tt} + 2s_t$}{}
\statement{3}{}{$R := R_{tt} - 1$}
\statement{4}{$temp\_borrow,R_o:=R_{tto}-1$}{return this $R_o$}
\statement{4}{$r:=r_{tto}-temp\_borrow$}{return this $r$}
\statement{2}{}{$S := S_t - 1$}
\statement{3}{$temp\_borrow,S_o:=S_{to}-1$}{}
\statement{3}{$s:=s_t-temp\_borrow$}{}
\statement{1}{else}{else if $R_t \ge 0$ $\iff$ else if $r_t \ge 0$}
\statement{2}{$R := R_t$}{}
\statement{2}{$S := S_t$}{}
\end{pseudocode}

- 6. correct return values:
    - 1\. if $R_t < 0$ then
        - 1. if $r_t < 0$ then
            - 1. $R := R_t + 2S_t - 1$
                - 1. $R_{tt} := R_t + 2S_t$
                    - $= r_t N + R_{to} + 2(s_t N + S_{to})$<br>
                      $= (r_t + 2s_t)N + R_{to} + 2S_{to}$
                    - 1. $r_{tto}' N + R_{tto} := R_{to} + 2S_{to}$
                    - 2. $r_{tto} := r_{tto}' + 2s_t$
                - 2. $R := R_{tt} - 1$
                    - $= r_{tt} N + R_{tto} -1$
                    - 1. $temp\_borrow, R_o := R_{tto} -1$
                    - 2. $r := r_{tto} - temp\_borrow$
            - 2.$S := S_t - 1$
    - 2. else
        - if $R_t \ge 0$
        - 1. $R := R_t$
        - 2. $S := S_t$
\end{algo}

\end{document}



\end{document}